\documentclass{HZNUMCM}
\usepackage{graphicx}
\usepackage{hyperref}
\usepackage{subcaption}
\usepackage{float}
\usepackage{svg}
\usepackage{mathrsfs}
\usepackage{fvextra}
\definecolor{customcolor}{HTML}{429938}

\setControlNumber{2511940}
\setContestType{MCM}
\setProblemLetter{E}
\setPaperTitle{Our Article}

%summary
\setSummary{ sumary sumary sumary sumary sumary sumary sumary sumary sumary sumary sumary sumary sumary sumary sumary sumary sumary sumary sumary sumary sumary sumary sumary sumary sumary sumary sumary sumary sumary sumary sumary sumary sumary sumary sumary sumary sumary sumary sumary sumary sumary sumary sumary sumary sumary sumary sumary sumary sumary sumary sumary sumary sumary sumary sumary sumary sumary sumary sumary sumary sumary sumary sumary sumary sumary sumary sumary sumary sumary sumary sumary sumary sumary sumary sumary sumary sumary sumary sumary sumary sumary sumary sumary sumary sumary sumary sumary sumary sumary sumary sumary sumary sumary sumary sumary sumary sumary sumary sumary sumary sumary sumary sumary sumary sumary sumary sumary sumary sumary sumary sumary sumary sumary sumary sumary sumary sumary sumary sumary sumary sumary sumary sumary sumary sumary sumary sumary sumary sumary sumary sumary sumary sumary sumary sumary sumary sumary sumary sumary sumary sumary sumary sumary sumary sumary sumary sumary sumary sumary sumary sumary sumary sumary sumary sumary sumary sumary sumary sumary sumary sumary sumary sumary sumary sumary sumary sumary sumary sumary sumary sumary sumary sumary sumary sumary sumary sumary sumary sumary sumary sumary sumary sumary sumary sumary sumary sumary sumary sumary sumary sumary sumary sumary sumary sumary sumary sumary sumary sumary sumary sumary sumary sumary sumary sumary sumary sumary sumary sumary sumary sumary sumary sumary sumary sumary sumary sumary sumary sumary sumary sumary sumary sumary sumary sumary sumary sumary sumary sumary sumary sumary sumary sumary sumary sumary sumary sumary sumary sumary sumary sumary sumary sumary sumary sumary sumary sumary sumary sumary sumary sumary sumary sumary sumary sumary sumary sumary sumary sumary sumary sumary sumary sumary sumary sumary sumary sumary sumary sumary sumary sumary sumary sumary sumary sumary sumary sumary sumary sumary sumary sumary sumary sumary sumary sumary sumary sumary sumary sumary sumary sumary sumary sumary sumary sumary sumary sumary sumary sumary sumary sumary sumary sumary sumary sumary sumary sumary sumary sumary sumary sumary sumary sumary sumary sumary sumary sumary sumary sumary}

%begin
\begin{document}
\showSummarySheet
\newpage % 添加此行
\showContents

  \section{Introduction}
    \subsection{Background}
    In the past few decades, with the rapid population growth, 
    food supply has become one of the most pressing global issues. 
    Under the conditions where scientific and technological advancements have not been fully adopted, 
    the existing arable land area is insufficient to meet the food demand. 
    As a result, many regions have resorted to deforestation for land conversion, 
    as shown in \figurename~\ref{fig:deforestation1}. \figurename~\ref{fig:deforestation2} illustrates that when the forest ecosystem is artificially disrupted, 
    its complex spatial structure is destroyed.
    \begin{figure}[H]
      \centering
        \begin{minipage}[b]{0.45\linewidth}
            \centering
            \includegraphics[height=4cm, keepaspectratio]{images/deforestation1.jpg} % 替换为你的第一张图片路径
            \caption{Deforestation for Farming}
            \label{fig:deforestation1}
        \end{minipage}
      \hspace{0.05\linewidth}
        \begin{minipage}[b]{0.45\linewidth}
            \centering
            \includegraphics[height=4cm, keepaspectratio]{images/deforestation2.jpg} % 替换为你的第二张图片路径
            \caption{Deforested Forest}
            \label{fig:deforestation2}
        \end{minipage}
      \end{figure}
    Therefore, the converted forest area must undergo a long process of ecological reconstruction. 
    However, due to the influence of human activities during this process, 
    the post-clearing forest ecosystem can no longer return to its original state, 
    but instead continuously undergoes succession into a stable agricultural ecosystem. 
    To achieve both economic and ecological benefits and fulfill sustainable development goals, 
    it is necessary to examine the ecological succession from converted forest to agricultural ecosystems and explore how green agriculture can bring dual benefits in terms of economic and ecological indicators.
      
    \subsection{Problem Analysis}
      The basic requirements of the project are to establish a model that reflects the ecological succession process from a converted forest area to a mature agricultural ecosystem over time. 
      The model should incorporate both natural processes and human agricultural activities. 
      Specifically, the requirements are as follows:
      \begin{itemize}
        \item \textbf{Develop an ecological model} for the converted forest area, particularly:
          \begin{itemize}
            \item \textbf{Food web construction}: The model should include at least producers (e.g., plants) and consumers (e.g., herbivores and predators) and their interactions.
            \item \textbf{Consideration of agricultural cycles and seasonality}: The impact of seasonal and cyclical agricultural activities should be taken into account.
            \item \textbf{Impact of chemicals}: The model should account for the effects of chemicals such as herbicides and pesticides on plants, insects, bats, birds, and the stability of the ecosystem.
          \end{itemize}
        \item \textbf{Reemergence of species during ecosystem maturation}: As the ecosystem matures, the model should consider the reemergence of two species and their impact on the ecosystem.
        \item \textbf{Impact of removing chemicals}: After the ecosystem matures, humans will attempt to remove chemicals. The model should assess the stability of the ecosystem after herbicides are removed, with the effects reflected through producers and consumers.
        \item \textbf{Introduction of bats into the food web}: The model should examine how bats, as insectivores and pollinators, interact with plants, insects, and predators, and how their inclusion influences the stability of the ecosystem. Additionally, the model should identify another species that could benefit the ecosystem and compare the effects of different species.
        \item \textbf{Analysis of the impact of organic farming}: The model should assess the impact of adopting organic farming practices, considering various scenarios and components. The evaluation should include the effects on the overall ecosystem and individual elements, such as pest control, crop health, plant reproduction, biodiversity, long-term sustainability, and cost-effectiveness. The model should analyze the impact of organic practices on pest management, soil health, and biodiversity, while weighing the economic costs and benefits. A comprehensive ecological and economic trade-off analysis should be provided to assess the feasibility of organic farming.
      \end{itemize}

    \subsection{Our Work}
    \begin{itemize}
      \item 1
      \item 2
      \item 3
    \end{itemize}

  \section{Assumptions and Notations}
    \subsection{Assumptions and Explanations}
      \begin{itemize}
        \item \textbf{Accurate Data Assumption}: The model assumes that the data used are accurate.\\
        \textbf{Explanation}: The data used in the model are sourced from official databases, and we believe the data to be accurate and reliable.
        
        \item \textbf{Geographic Applicability Assumption}: The model assumes that the applicable region is Southeast Asia,
         where two crops of rice are planted each year in the farmland.\\
        \textbf{Explanation}: The climate of Southeast Asia is rather simple, 
        with only two seasons-rainy and dry. Additionally, as is shown in \figurename~\ref{fig:Temperature},
        the temperature variation within a year is minimal, which has a trivial effect on the ecosystem.
        Consequently, temperature can be considered as a constant.
        Due to such weather pattern, it aligns with the planting patterns commonly observed in Southeast Asia to plant three crops of rice each year(showed in \figurename~\ref{fig:PlantMode}),
         and the simplicity of crop types makes the model easier to establish.
        
        \begin{figure}[H]
          \centering
          \includegraphics[width=\linewidth]{images/AverTemper.png}
          \caption{Temperature Data in Southeast Asia\cite{IndoTemper}}
          \label{fig:Temperature}
        \end{figure}

        \begin{figure}[H]
          \centering
          \includegraphics[width=0.75\linewidth]{images/PlantMode.jpg}
          \caption{Agricultural Cycle in Southeast Asia\cite{IndoRice}}
          \label{fig:PlantMode}
        \end{figure}

        \item \textbf{Stable Trait Assumption}: The model assumes that the traits of all organisms remain stable.\\
        \textbf{Explanation}:Since the time span considered in the model is much shorter than the time required for evolutionary changes or mutations to occur,
         the traits of organisms are assumed to remain stable. This assumption also helps simplify the model.
        
        \item \textbf{Stable Lighting Conditions Assumption}: The model assumes that the region under study experiences stable lighting conditions throughout the four seasons.\\
        \textbf{Explanation}: Since the model focuses on tropical regions, the variation in daylight duration across different months within a year is minimal,
         thus the lighting conditions are treated as constant in the model.
        
        \item \textbf{Stable Growth Environment Assumption}: The model assumes that no natural disasters,
         which could significantly impact the agricultural ecosystem, will occur during the time frame considered.\\
        \textbf{Explanation}: Natural disasters are considered low-probability events in agricultural activities.
         To ensure the generalizability of the model, natural disasters should not be considered.
      \end{itemize}

        \subsection{Notations}
      % table
      \begin{table}[H]
        \centering
        \caption{Notations}
        \begin{tabular}{cc}
          \toprule
          \rowcolor{customcolor!40} % 设置背景颜色
          Symbols & Description\\
          \midrule
          $\mathbf{X}$ & Vector $[N_{wd},N_{crp},N_{pst},N_{ins},...,C_{hc},C_{pc}]^T$ to describe the system,etc. \\
          $wd$ & Subscription for weeds \\
          $crp$ & Subscription for crops \\
          $stw$ & Subscription for straw \\
          $pst$ & Subscription for pest(who consumes crops) \\
          $ins$ & Subscription for other insects(who consume weeds) \\
          $bd$ & Subscription for small birds(herbivorous) \\
          $Bd$ & Subscription for huge birds(carnivorous) \\
          $bt$ & Subscription for bats \\
          $snk$ & Subscription for snake \\
          $frg$ & Subscription for frog \\
          $HC$ & Subscription for herbicide \\
          $PC$ & Subscription for pesticide \\
          $C_i$ & Concentration of certain chemical \\
          $N_i$ & Numbers of certain species \\
          $W_i$ & Density of biomass of certain species \\
          $w_i$ & Average mass density of individuals of certain individuals \\
          $r_i$ & Natural growth gate of certain population\\
          $\mathscr{K}_i$ & Carrying capacity of certain population\\
          $\alpha$ & The effect of chemical concentration on growth rate\\
          $\beta_{i \rightarrow j}$ & Interspecific competition factor\\
          $\gamma$ & Activity of decomposer\\
          $A_i,B_i$ & Effect of the predator-prey relationship on population $i$ (specified coefficient).\\
          $D_i,E_i$ & Effect of shortage of food on population $i$ (specified coefficient)\\
          $N,P,K$ & Chemical elements\\
          \bottomrule
        \end{tabular}
        \label{tab:Notations}
      \end{table}

  \section{Models}
    \subsection{Model Development Explanation and Sequence}
      \subsubsection{Notes for Models}
        Before building all the models, here are some notes for all the models.
        
        First, biomass is chosen over population density. This is because rice, 
        as the primary producer in the agricultural ecosystem, 
        has its population density artificially determined (i.e., it does not reproduce). 
        Only the changes in its total biomass can reflect the developmental trend of the rice population.
        
        Second, regarding interspecific competition, since the agricultural ecosystem is relatively simple compared to others, 
        for the sake of model simplification, only interspecific competition between rice and weeds is considered. 

        Next, for the purpose of further simplifying the model, some parameters are combined. 
        For example, the predation rate is incorporated into the predation coefficient of the prey species, denoted as \(A\).
        
        Finally, since consumers are in a natural reproductive state with a relatively fixed age structure, 
        the average individual biomass can be assumed to be constant. 
        Therefore, biomass is directly proportional to the number of individuals in the population.
        
        It should be noted that at the start time, all the variations and coefficients mentioned above are positive real numbers.
      \subsubsection{Sequence}  
        Based on the conditions outlined above, we will gradually develop the model. 
        Given the numerous factors involved in the final model, we adopt a step-by-step approach, 
        starting from the simplest and progressing to more complex structures.
        
        Initially, we will construct Model I: LVLG Model, based on the Lotka-Volterra model and the Leslie-Gower model, 
        under the simplest food web conditions. This model will consider only producers, primary consumers, and secondary consumers.

        Subsequently, as the ecosystem gradually develops, we will build Model II. 
        In this model, we will incorporate the agricultural cycle (three crops per year), 
        seasonal rotation (rainy and dry seasons), and the use of chemicals (herbicides and pesticides).
        
        Next, we will consider the re-colonization of species and develop a more complex food web to form Model III. 
        In this model, we will retain the features of Model II while considering the intricate interactions between multiple species and between species and the environment.

        Finally, after the agricultural ecosystem matures, we will introduce human agricultural decisions, 
        such as the removal of pesticides and the intentional introduction of new populations, to construct Model IV.
    
    \subsection{Model I: LVLG(Lotka-Volterra and Leslie-Gower) Model}
      We will first discuss the initial conditions of the model from the perspective of biological populations. 
      The vertical structure of tropical forests is typically divided into several layers, including the canopy layer, understory, shrub layer, and ground layer. 
      Each layer not only supports different plant species but also provides habitat and food sources for various animals. 
      
      Deforestation will severely disrupt the vertical structure of the tropical rainforest food web. 
      Therefore, our model assumes that at the initial time point following deforestation, 
      the vertical structure of the ecosystem above the ground retains only a portion of the ground layer. 
      All populations that previously depended on the canopy, understory, and shrub layers for habitat and food sources will have migrated out of the ecosystem.

      Based on this assumption, the agricultural ecosystem in its initial food web retains only the following populations: plants, insects, and birds. 
      First, to preserve interspecific competition and align with reality as closely as possible, we divide plants into two populations: rice and weeds. 
      Second, although insects may exhibit food preferences, for simplicity, we assume that there is only one insect species that feeds on both crops and weeds. 
      Finally, based on the feeding habits of birds in real life, we assume that birds feed on both insects and the two plant species. 
      \figurename~\ref{fig:LGVGField} presents a schematic diagram of the ecosystem, and \figurename~\ref{fig:SimpleFoodWeb} shows the initial food web of the ecosystem.

      \begin{figure}[H]
        \centering
          \begin{minipage}[b]{0.45\linewidth}
              \centering
              \includegraphics[height=5cm, keepaspectratio]{images/LGVGField.pdf} % 替换为你的第一张图片路径
              \caption{Schematic map for LGVG Model}
              \label{fig:LGVGField}
          \end{minipage}
        \hspace{0.05\linewidth}
          \begin{minipage}[b]{0.45\linewidth}
              \centering
              \includegraphics[height=5cm, keepaspectratio]{images/SimpleFoodWeb.pdf}
              \caption{Food web in LGVG Model}
              \label{fig:SimpleFoodWeb}
          \end{minipage}
        \end{figure}
      Set February-when rice has just been planted-as the start time.
      Around the start time, in the simplest case, when climate, soil, and other conditions are favorable, 
      only biological factors should be considered.
      If the population sizes of producers and primary consumers are used to describe the entire system, 
      the Lotka-Volterra Model\cite{wangersky1978lotka} and Leslie-Gower Model\cite{GUO20142850} can be applied as follows:
      \begin{equation}
        \begin{aligned}
          \frac{\mathrm{d}W_{crp}}{\mathrm{d}t}&=r_{crp}W_{crp}\left( 1-\frac{W_{crp}+\beta _{w\rightarrow c}W_{wd}}{\mathscr{K} _{crp}} \right) -\frac{A_{crp}W_{crp}W_{ins}}{1+B_{crp}W_{crp}}\\
          \frac{\mathrm{d}W_{wd}}{\mathrm{d}t}&=r_{wd}W_{wd}\left( 1-\frac{W_{wd}+\beta _{c\rightarrow w}W_{crp}}{\mathscr{K} _{wd}} \right) -\frac{A_{wd}W_{wd}W_{ins}}{1+B_{wd}W_{wd}}\\
          \frac{\mathrm{d}W_{ins}}{\mathrm{d}t}&=r_{ins}W_{ins}\left[ 1-\frac{D_{ins}W_{ins}}{1+E_{ins}\left( 0.6W_{crp}+0.4W_{wd} \right)} \right] -\frac{A_{ins}W_{ins}W_{bd}}{1+B_{ins}W_{ins}}\\
          \frac{\mathrm{d}W_{bd}}{\mathrm{d}t} &= r_{bd}W_{bd} \left[ 1 - \frac{D_{bd}W_{bd}}{1 + E_{bd}(0.2W_{ins} + 0.2W_{wd} + 0.6W_{crp})} \right]\\
        \end{aligned} 
      \end{equation}
      In general, these four equations introduce the natural growth term 
      \begin{equation}
        r_i W_i
      \end{equation}
      for population growth. 
      The first two equations incorporate the interspecific competition term 
      \begin{equation}
        \beta_{i \rightarrow j} W_i,
      \end{equation}
      the larger this term, the more intense the interspecific competition, and the slower the biomass growth rate of population $i$. 
      The first three equations consider the effect of predation through the term(e.g. the first one) 
      \begin{equation}
        \frac{A_{crp} W_{crp} W_{ins}}{1 + B_{crp} W_{crp}}. 
      \end{equation}
      The last two equations(e.g. the third one) include the term 
      \begin{equation}
        \frac{D_{ins} W_{ins}}{1 + E_{ins} \left( 0.6 W_{crp} + 0.4 W_{wd} \right)},
      \end{equation}
      which represents biomass reduction due to food scarcity. 
      In the denominator, one factor is weighted according to the predation ratio: the less food available, 
      the more predators there are, resulting in a slower growth rate of the predator population.

      
    \subsection{Model II: CAGED(Conprehensive Agro-Ecosystem Dynamics) Model}
      Model I is quite simple and aims to construct a basic short-term model. 
      Considering the impact of long-term environmental factors, 
      in order to better simulate the long-term evolutionary behavior of the ecosystem model, 
      we will progressively introduce the effects of simplified agricultural cycles, 
      seasonality, soil fertility, and random factors on the system dynamics model. 
      Furthermore, we will consider the impact of chemical use, 
      specifically the application of herbicides and pesticides, on the system, 
      building upon this natural agricultural ecosystem model.

      \subsubsection{Agricultural Cycle}
        According to rice production in Indonesia(\figurename~\ref{fig:PlantModePlus}), 
        rice production in Southeast Asia follows a three-crop-per-year pattern. 
        We assume that February, June, and October are the overlapping periods of two agricultural cycles 
        (from the last post-harvest period to the new planting period).
        \begin{figure}[H]
          \centering
          \includegraphics[width=0.7\linewidth]{images/PlantModePlus.png}
          \caption{Overlapping Periods of Two Agricultural Cycles}
          \label{fig:PlantModePlus}
        \end{figure}
        During the overlapping period, the rice population re-enters the food web in the form of seeds after harvesting, 
        starting a new agricultural cycle. The mature rice straw, as an agricultural byproduct, 
        remains in the ecosystem after harvesting, and after being treated by methods such as burning or returning to the field, 
        the rice biomass is decomposed and ingested by decomposers and insects, ultimately decreasing to zero.

        Based on literature review\cite{OLIVER20191139,summers2003biomass}, 
        the biomass fate of mature rice during each overlapping period can be uniformly described by \figurename~\ref{fig:rice_to}. 
        \begin{figure}[H]
          \centering
          \includegraphics[width=\linewidth]{images/rice_to.pdf}
          \caption{Biomass Fate of Mature Rice during Each Overlapping Period}
          \label{fig:rice_to}
        \end{figure}
        Since decomposers' biomass is not considered in our food web model, 
        we only analyze the impact of decomposed rice straw residue(the 36\% part) on the producers—namely rice and weeds—in the new cycle. 
        For further details, refer to the soil fertility section later in the paper.
      
        For simplification of the model, 
        the biomass change of the rice population during each overlapping period is represented 
        by a biomass step function at specific time points in the mathematical model (i.e., the system of difference equations). 
        Since $t=0$ corresponds to the start of the first planting season, with one year assumed to have 52 weeks, 
        and each week being a time step, based on the rice production pattern \figurename~\ref{fig:PlantModePlus}, 
        we define the step moments at week 0, week 17, and week 34 of each year. The biomass of rice at these times follows:
      
        \begin{equation}
        W_{crp}|_{t=n}=0.1W_{crp}|_{t=n-1}, \quad n = 52k, 52k+17, 52k+34, \quad (year\, k \geqslant 0, \, week\, n \geqslant 1)
        \end{equation}
      
        When considering the impact of rice straw on insect biomass, 
        we treat rice straw as the 'prey' of insects and use the modified Lotka-Volterra (LG) predator-prey model to characterize this effect. 
        At each step moment, rice straw enters the food web model with an initial biomass of $0.09W_{crp}$, and $r_{stw} = 0$. 
        Thus, the rice straw-insect model is established as:
        
        \begin{equation}
          \frac{\mathrm{d}W_{stw}}{\mathrm{d}t} = \frac{A_{stw} W_{stw} W_{ins}}{1 + B_{stw} W_{stw}} \quad (t \neq n)
        \end{equation}
        \begin{equation}
          W_{stw}|_{t=n} = W_{stw}|_{t=n-1} + 0.09 W_{crp}|_{t=n-1},\\
          n = 52k, 52k+17, 52k+34, \quad (k \geqslant 0, \, n \geqslant 1)
        \end{equation}
        
        The difference equation for insect biomass is modified as follows:
        
        \begin{equation}
          \frac{\mathrm{d}W_{ins}}{\mathrm{d}t} = r_{ins} W_{ins} \left[ 1 - \frac{D_{ins} W_{ins}}{1 + E_{ins} \left( 0.6 W_{crp} + 0.4 W_{wd} + W_{stw} \right)} \right] - \frac{A_{ins} W_{ins} W_{bd}}{1 + B_{ins} W_{ins}}
        \end{equation}
      
      \subsubsection{Seasonality}
        The seasonal influences primarily include factors such as light, 
        climate (temperature, precipitation), and biological habits, 
        all of which have significant direct effects on both $r$ and $\mathscr{K}$ within populations. 
        Considering that many of these factors are difficult to quantify precisely, 
        we set sinusoidal perturbations in the periodic parameter $p(T)$ as follows\cite{GAKKHAR20061239}:

        \begin{equation}
          p(t) = \bar{p} \left[ 1 + \epsilon \sin (\Omega t + \phi) \right]
        \end{equation}
        
        Specifically, in the LV-LG model mentioned above, 
        for population $i$, the values of $r_i$ and $K_i$ are defined as follows:
        
        \begin{align}
          r_i &= \bar{r}_i \left[ 1 + \epsilon_i \sin \left( \Omega_i t + \phi_i \right) \right] \\
          \mathscr{K}_i &= \bar{\mathscr{K}}_i \left[ 1 + \epsilon_i \sin \left( \Omega_i t + \phi_i' \right) \right], \quad \left( i = \text{crp or wd} \right) \\
          D_i &= \frac{\bar{D}_i'}{1 + \epsilon_i \sin \left( \Omega_i t + \phi_i' \right)}, \quad \left( i \neq \text{crp or wd} \right)
        \end{align}
        
        Where $\bar{r}_i$ and $\bar{\mathscr{K}}_i$ are the corresponding periodic means, 
        which can be treated as constants; $\Omega_i$ is the seasonal fluctuation angular frequency, 
        and $T = \frac{2\pi}{\Omega}$ is the seasonal fluctuation period; 
        $\epsilon_i$ is the seasonal impact parameter. Since $r_i, \mathscr{K}_i \geqslant 0$, 
        it follows that $-1 \leqslant \epsilon_i \leqslant 1$; $\phi_i$ and $\phi_i'$ are the phases, 
        with $0 \leqslant \phi_i < 2\pi$, which characterize the asynchronous fluctuations of biomass in different species.
        
        It can be mathematically proven that, after incorporating these variations, 
        the LV-LG model system theoretically possesses non-trivial stable points when the parameters are within a certain range. 
        When the system is near these points, the ecosystem can recover from small perturbations, 
        reaching a dynamic ecological equilibrium.

      \subsubsection{Soil Fertility}
        The impact of soil fertility on plant growth is introduced. For simplicity in the model, 
        the effect of soil fertility on plant growth is divided into 
        the influence of the concentration of inorganic salts absorbed by the plants and the concentration of carbon dioxide.

        For the concentration of inorganic salts absorbed by the plants, 
        it is closely related to fertilizer concentration (affected by straw and chemical fertilizers), 
        decomposer activity, and the plant's growth stage. Taking rice as an example, during the seedling stage, 
        the concentration of inorganic salts in the plant is relatively low, with the rate of increase being positive; 
        from the tillering to jointing stage, the demand for inorganic salts rapidly increases, 
        and the concentration reaches a turning point; from the booting to flowering stage, 
        the rate of increase slows down until the concentration reaches its maximum; 
        in the milk to yellow ripening stage, the absorption of inorganic salts decreases, 
        and the concentration declines.
        %\cite{ref}.

        We use a Gaussian-like probability density function to fit the trend of inorganic salt concentration in the plant over each agricultural cycle, 
        expressed as:

        \begin{equation}
        C_{isc}(t) = \alpha_{isc} e^{-\frac{(t - T_{mc})^2}{\sigma_{isc}}}
        \end{equation}
        \begin{equation}
        C_{isw}(t) = \alpha_{isw} e^{-\frac{(t - T_{mw})^2}{\sigma_{isw}}}
        \end{equation}
        where $T_{mc}, T_{mw}$ represent the moment when the concentration of inorganic salts in the rice or weed reaches its maximum during a certain cycle; 
        $\alpha_{isc}, \alpha_{isw},  \sigma_{isc}, \sigma_{isw}$ correspond to the model parameters for rice or weeds, 
        which are determined by the fertilizer concentration (chemical fertilizers and straw), decomposer activity, and the plant's growth stage.
      \subsubsection{Random Factor}
      \subsubsection{Chemical Use}
        Considering that the application of herbicides during rice cultivation mainly occurs before sowing, 
        during the seedling stage, and the tillering stage, 
        we assume that herbicides are applied three times during an agricultural cycle, 
        with time points at week 0, week 3, and week 8. The herbicide application concentration is proportional to the current weed biomass. 
        Since the herbicide application frequency is low in this case, 
        we can neglect the specific negative effects of herbicides on rice (any minimal negative impact can be covered by the random process in Section 3.3.4).

        We consider weeds as the 'prey' of the herbicide. 
        Similar to the previous straw-insect model, 
        we can use a modified Lotka-Volterra predator-prey model, 
        where \( r_n \equiv 0 \).

        For the degradation of herbicide, we can use an exponential decay model. 
        This leads to the establishment of the weed-herbicide model:

        \begin{equation}
        C_{hc}|_{t=m} = C_{hc}|_{t=m-1} + \alpha W_{wd}|_{t=m}
        \end{equation}

        \begin{equation}
        \frac{W_{wd}}{dt} = r_{wd}W_{wd}\left(1 - \frac{W_{wd} + \beta_{c \rightarrow w} W_{crp}}{\mathscr{K}_{wd}}\right) - \frac{A_{wd}W_{wd}W_{ins}}{1 + \beta_{wd}W_{wd}} - \frac{A_{hc}W_{wd}C_{hc}}{1 + B_{hc}W_{wd}}
        \end{equation}
        
        \begin{equation}
        \frac{dC_{hc}}{dt} = -\alpha_{d}C_{hc}
        \end{equation}

    \subsection{Model III: }
      \begin{figure}[H]
        \centering
        \includegraphics[width=0.5\linewidth]{images/food_web.pdf}
        \caption{Food web for NGLG Model}
        \label{fig:NGLG_FoodWeb}
      \end{figure}
    \subsection{Model IV: }
  \section{Application of the Models}

  \section{Sensitivity Analysis}

  \section{Evaluation of the Model}
    \subsection{Strengths}
    \subsection{Weaknesses}

  \section{Conclusion}

  %%citation
  % as \figurename~\ref{fig:image1} shows,this is a picture.
  % ...\cite{example1}
  % 123123123\cite{rosenow1983drought}

  \addcontentsline{toc}{section}{References}
  \bibliographystyle{unsrt}%{brief}%{alpha}%{unsrt}
  \bibliography{article_file}

\end{document}