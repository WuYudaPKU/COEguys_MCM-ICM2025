\documentclass{HZNUMCM}
\usepackage{graphicx}
\usepackage{hyperref}
\usepackage{subcaption}
\definecolor{customcolor}{HTML}{009EAE}

\setControlNumber{888888}
\setContestType{MCM}
\setProblemLetter{A}
\setPaperTitle{Community Succession Simulation: Surviving Drought}

%summary
\setSummary{ sumary sumary sumary sumary sumary sumary sumary sumary sumary sumary sumary sumary sumary sumary sumary sumary sumary sumary sumary sumary sumary sumary sumary sumary sumary sumary sumary sumary sumary sumary sumary sumary sumary sumary sumary sumary sumary sumary sumary sumary sumary sumary sumary sumary sumary sumary sumary sumary sumary sumary sumary sumary sumary sumary sumary sumary sumary sumary sumary sumary sumary sumary sumary sumary sumary sumary sumary sumary sumary sumary sumary sumary sumary sumary sumary sumary sumary sumary sumary sumary sumary sumary sumary sumary sumary sumary sumary sumary sumary sumary sumary sumary sumary sumary sumary sumary sumary sumary sumary sumary sumary sumary sumary sumary sumary sumary sumary sumary sumary sumary sumary sumary sumary sumary sumary sumary sumary sumary sumary sumary sumary sumary sumary sumary sumary sumary sumary sumary sumary sumary sumary sumary sumary sumary sumary sumary sumary sumary sumary sumary sumary sumary sumary sumary sumary sumary sumary sumary sumary sumary sumary sumary sumary sumary sumary sumary sumary sumary sumary sumary sumary sumary sumary sumary sumary sumary sumary sumary sumary sumary sumary sumary sumary sumary sumary sumary sumary sumary sumary sumary sumary sumary sumary sumary sumary sumary sumary sumary sumary sumary sumary sumary sumary sumary sumary sumary sumary sumary sumary sumary sumary sumary sumary sumary sumary sumary sumary sumary sumary sumary sumary sumary sumary sumary sumary sumary sumary sumary sumary sumary sumary sumary sumary sumary sumary sumary sumary sumary sumary sumary sumary sumary sumary sumary sumary sumary sumary sumary sumary sumary sumary sumary sumary sumary sumary sumary sumary sumary sumary sumary sumary sumary sumary sumary sumary sumary sumary sumary sumary sumary sumary sumary sumary sumary sumary sumary sumary sumary sumary sumary sumary sumary sumary sumary sumary sumary sumary sumary sumary sumary sumary sumary sumary sumary sumary sumary sumary sumary sumary sumary sumary sumary sumary sumary sumary sumary sumary sumary sumary sumary sumary sumary sumary sumary sumary sumary sumary sumary sumary sumary sumary sumary sumary sumary sumary sumary sumary sumary sumary}

%begin
\begin{document}
\showSummarySheet
\showContents

  \section{Introduction}
    \subsection{Background}
    \subsection{Restatement of the Problem}
    \subsection{Our Work}
    \begin{itemize}
      \item work1
      \item work2
      \item work3
    \end{itemize}

  \section{Assumptions and Notation}
    \subsection{Assumptions}
    \subsection{Notation}

  \section{Model}

  \section{Application of the Model}

  \section{Sensitivity Analysis}

  \section{Evaluation of the Model}
    \subsection{Strengths}
    \subsection{Weaknesses}

  \section{Conclusion}
  % figure
  \begin{figure}[ht]
    \centering
    \includegraphics[width=0.5\linewidth]{images/peaks.png} % 替换为你的第一张图片路径
    \caption{peaks for test}
    \label{fig:image1}
  \end{figure}

  \begin{figure}[ht]
      \centering
      \begin{minipage}[b]{0.45\linewidth}
          \centering
          \includegraphics[height=5cm, keepaspectratio]{images/peaks.png} % 替换为你的第一张图片路径
          \caption{First Image}
          \label{fig:image2}
      \end{minipage}
      \hspace{0.05\linewidth}
      \begin{minipage}[b]{0.45\linewidth}
          \centering
          \includegraphics[height=5cm, keepaspectratio]{images/courses.png} % 替换为你的第二张图片路径
          \caption{Second Image}
          \label{fig:image3}
      \end{minipage}
  \end{figure}
  
  as \figurename~\ref{fig:image1} shows,this is a picture.

  ...\cite{example1}
  123123123\cite{rosenow1983drought}

  % table
  \begin{table}[h]
    \centering
    \caption{An example of a three-line table.}
    \begin{tabular}{lccc}
      \toprule
      \rowcolor{customcolor!50} % 设置背景颜色为浅灰色
      Column 1 & Column 2 & Column 3 & Column 4 \\
      \midrule
      Data 1 & Data 2 & Data 3 & Data 4 \\
      Data 4 & Data 5 & Data 6 & Data 8 \\
      \bottomrule
    \end{tabular}
    \label{tab:example}
  \end{table}

  %lst
  \addcontentsline{toc}{section}{References}
  \bibliographystyle{unsrt}%{brief}%{alpha}%{unsrt}
  \bibliography{article_file}

\end{document}