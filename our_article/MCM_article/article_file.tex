\documentclass{HZNUMCM}
\usepackage{graphicx}
\usepackage{hyperref}
\usepackage{subcaption}
\usepackage{float}
\definecolor{customcolor}{HTML}{429938}

\setControlNumber{2511940}
\setContestType{MCM}
\setProblemLetter{E}
\setPaperTitle{Our Article}

%summary
\setSummary{ sumary sumary sumary sumary sumary sumary sumary sumary sumary sumary sumary sumary sumary sumary sumary sumary sumary sumary sumary sumary sumary sumary sumary sumary sumary sumary sumary sumary sumary sumary sumary sumary sumary sumary sumary sumary sumary sumary sumary sumary sumary sumary sumary sumary sumary sumary sumary sumary sumary sumary sumary sumary sumary sumary sumary sumary sumary sumary sumary sumary sumary sumary sumary sumary sumary sumary sumary sumary sumary sumary sumary sumary sumary sumary sumary sumary sumary sumary sumary sumary sumary sumary sumary sumary sumary sumary sumary sumary sumary sumary sumary sumary sumary sumary sumary sumary sumary sumary sumary sumary sumary sumary sumary sumary sumary sumary sumary sumary sumary sumary sumary sumary sumary sumary sumary sumary sumary sumary sumary sumary sumary sumary sumary sumary sumary sumary sumary sumary sumary sumary sumary sumary sumary sumary sumary sumary sumary sumary sumary sumary sumary sumary sumary sumary sumary sumary sumary sumary sumary sumary sumary sumary sumary sumary sumary sumary sumary sumary sumary sumary sumary sumary sumary sumary sumary sumary sumary sumary sumary sumary sumary sumary sumary sumary sumary sumary sumary sumary sumary sumary sumary sumary sumary sumary sumary sumary sumary sumary sumary sumary sumary sumary sumary sumary sumary sumary sumary sumary sumary sumary sumary sumary sumary sumary sumary sumary sumary sumary sumary sumary sumary sumary sumary sumary sumary sumary sumary sumary sumary sumary sumary sumary sumary sumary sumary sumary sumary sumary sumary sumary sumary sumary sumary sumary sumary sumary sumary sumary sumary sumary sumary sumary sumary sumary sumary sumary sumary sumary sumary sumary sumary sumary sumary sumary sumary sumary sumary sumary sumary sumary sumary sumary sumary sumary sumary sumary sumary sumary sumary sumary sumary sumary sumary sumary sumary sumary sumary sumary sumary sumary sumary sumary sumary sumary sumary sumary sumary sumary sumary sumary sumary sumary sumary sumary sumary sumary sumary sumary sumary sumary sumary sumary sumary sumary sumary sumary sumary sumary sumary sumary sumary sumary sumary sumary sumary sumary sumary sumary sumary}

%begin
\begin{document}
\showSummarySheet
\showContents

  \section{Introduction}
    \subsection{Background}
      \begin{figure}[ht]
      \centering
        \begin{minipage}[b]{0.45\linewidth}
            \centering
            \includegraphics[height=4cm, keepaspectratio]{images/deforestation1.jpg} % 替换为你的第一张图片路径
            \caption{Deforestation for Farming}
            \label{fig:deforestation1}
        \end{minipage}
      \hspace{0.05\linewidth}
        \begin{minipage}[b]{0.45\linewidth}
            \centering
            \includegraphics[height=4cm, keepaspectratio]{images/deforestation2.jpg} % 替换为你的第二张图片路径
            \caption{Deforested Forest}
            \label{fig:deforestation2}
        \end{minipage}
      \end{figure}
    \subsection{Problem Analysis}
    \subsection{Our Work}
    \begin{itemize}
      \item 1
      \item 2
      \item 3
    \end{itemize}

  \section{Assumptions and Notations}
    \subsection{Assumptions and Explanations}
      \begin{itemize}
        \item \textbf{Accurate Data Assumption}: The model assumes that the data used are accurate.

        \textbf{Explanation}: The data used in the model are sourced from official databases, and we believe the data to be accurate and reliable.
        \item \textbf{Geographic Applicability Assumption}: The model assumes that the applicable region is Southeast Asia.

        \textbf{Explanation}: The climate of Southeast Asia is simple, with only two seasons—rainy and dry. Additionally, as is shown in \figurename~\ref{fig:Temperature},the temperature variation within a year is minimal,which leads to trivial effect on the ecosystem.Consequently,temperature can be considered as a constant.
        \begin{figure}[H]
          \centering
          \includegraphics[width=\linewidth]{images/AverTemper.png}
          \caption{Mean Temperature from 1991 to 2020 in Southeast Asia}
          \label{fig:Temperature}
        \end{figure}
        \item \textbf{Planting Pattern Assumption}: The model assumes that two crops of rice are planted each year in the farmland.
  
        \textbf{Explanation}: This aligns with the planting patterns commonly observed in Southeast Asia, and the simplicity of crop types makes the model easier to establish.
        \item \textbf{Stable Lighting Conditions Assumption}: The model assumes that the region under study experiences stable lighting conditions throughout the four seasons.
  
        \textbf{Explanation}: Since the model focuses on tropical regions, the variation in daylight duration across different months within a year is minimal, thus the lighting conditions are treated as constant in the model.
        \item \textbf{Stable Growth Environment Assumption}: The model assumes that no natural disasters, which could significantly impact the agricultural ecosystem, will occur during the time frame considered.
  
        \textbf{Explanation}: Natural disasters are considered low-probability events in agricultural activities. To ensure the generalizability of the model, natural disasters should not be considered.
      \end{itemize}
        \subsection{Notations}
      % table
      \begin{table}[h]
        \centering
        % \caption{An example of a three-line table.}
        \begin{tabular}{cc}
          \toprule
          \rowcolor{customcolor!40} % 设置背景颜色
          Symbols & Description\\
          \midrule
          $\mathbf{X}$ & Vector $[N_w,N_c,N_p,N_b,N_B,C_{hc},C_{pc}]^T$,etc. \\
          $w$ & Subscription for weeds \\
          $c$ & Subscription for crops \\
          $p$ & Subscription for pest \\
          $bir$ & Subscription for birds \\
          $bat$ & Subscription for bats \\
          $hc$ & Subscription for herbicide \\
          $pc$ & Subscription for pesticide \\
          $C_i$ & Concentration of certain chemical \\
          $N_i$ & Numbers of certain species \\
          $\alpha$ & $abc$ \\
          \bottomrule
        \end{tabular}
        \label{tab:example}
      \end{table}

  \section{Models}
        % figure
    \begin{figure}[ht]
      \centering
      \includegraphics[width=0.5\linewidth]{images/energy_flow.png} % 替换为你的第一张图片路径
      \caption{Energy Flow}
      \label{fig:EnergyFlow}
    \end{figure}

  \section{Application of the Models}

  \section{Sensitivity Analysis}

  \section{Evaluation of the Model}
    \subsection{Strengths}
    \subsection{Weaknesses}

  \section{Conclusion}

  % % figure
  % \begin{figure}[ht]
  %   \centering
  %   \includegraphics[width=\linewidth]{images/scatter.pdf} % 替换为你的第一张图片路径
  %   \caption{the scatter}
  %   \label{fig:image1}
  % \end{figure}

  % \begin{figure}[ht]
  %     \centering
  %     \begin{minipage}[b]{0.45\linewidth}
  %         \centering
  %         \includegraphics[height=5cm, keepaspectratio]{images/peaks.png} % 替换为你的第一张图片路径
  %         \caption{First Image}
  %         \label{fig:image2}
  %     \end{minipage}
  %     \hspace{0.05\linewidth}
  %     \begin{minipage}[b]{0.45\linewidth}
  %         \centering
  %         \includegraphics[height=5cm, keepaspectratio]{images/courses.png} % 替换为你的第二张图片路径
  %         \caption{Second Image}
  %         \label{fig:image3}
  %     \end{minipage}
  % \end{figure}

  %%citation
  % as \figurename~\ref{fig:image1} shows,this is a picture.
  % ...\cite{example1}
  % 123123123\cite{rosenow1983drought}

  % % table
  % \begin{table}[h]
  %   \centering
  %   \caption{An example of a three-line table.}
  %   \begin{tabular}{lccc}
  %     \toprule
  %     \rowcolor{customcolor!50} % 设置背景颜色
  %     Column 1 & Column 2 & Column 3 & Column 4 \\
  %     \midrule
  %     Data 1 & Data 2 & Data 3 & Data 4 \\
  %     Data 4 & Data 5 & Data 6 & Data 8 \\
  %     \bottomrule
  %   \end{tabular}
  %   \label{tab:example}
  % \end{table}

  \addcontentsline{toc}{section}{References}
  \bibliographystyle{unsrt}%{brief}%{alpha}%{unsrt}
  \bibliography{article_file}

\end{document}